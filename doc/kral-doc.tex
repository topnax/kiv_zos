\documentclass[12pt, a4paper]{article}

\usepackage[czech]{babel}
\usepackage{lmodern}
\usepackage[utf8]{inputenc}
\usepackage[T1]{fontenc}
\usepackage{graphicx}
\usepackage{amsmath}
\usepackage[hidelinks,unicode]{hyperref}
\usepackage{float}
\usepackage{listings}
\usepackage{tikz}
\usepackage{xcolor}
\usepackage[final]{pdfpages}
\usepackage{tabularx}

\definecolor{mauve}{rgb}{0.58,0,0.82}
\usetikzlibrary{shapes,positioning,matrix,arrows}

\newcommand{\img}[1]{(viz obr. \ref{#1})}

\definecolor{pblue}{rgb}{0.13,0.13,1}
\definecolor{pgreen}{rgb}{0,0.5,0}
\definecolor{pred}{rgb}{0.9,0,0}
\definecolor{pgrey}{rgb}{0.46,0.45,0.48}

\lstset{frame=tb,
  language=C,
  aboveskip=3mm,
  belowskip=3mm,
  showstringspaces=false,
  columns=flexible,
  basicstyle={\small\ttfamily},
  numbers=none,
  numberstyle=\tiny\color{gray},
  keywordstyle=\color{blue},
  commentstyle=\color{dkgreen},
  stringstyle=\color{mauve},
  breaklines=true,
  breakatwhitespace=true,
  tabsize=3
}

\lstset{language=Java,
  showspaces=false,
  showtabs=false,
  breaklines=true,
  showstringspaces=false,
  breakatwhitespace=true,
  commentstyle=\color{pgreen},
  keywordstyle=\color{pblue},
  stringstyle=\color{pred},
  basicstyle=\ttfamily,
  moredelim=[il][\textcolor{pgrey}]{$$},
  moredelim=[is][\textcolor{pgrey}]{\%\%}{\%\%}
}

\let\oldsection\section
\renewcommand\section{\clearpage\oldsection}

\begin{document}
	% this has to be placed here, after document has been created
	% \counterwithout{lstlisting}{chapter}
	\renewcommand{\lstlistingname}{Ukázka zprávy}
	\renewcommand{\lstlistlistingname}{Seznam ukázek}
    \begin{titlepage}

       \centering

       \vspace*{\baselineskip}

       \begin{figure}[H]
          \centering
          \includegraphics[width=7cm]{img/fav-logo.jpg}
       \end{figure}

       \vspace*{1\baselineskip}
       {\sc Semestrální práce z předmětu KIV/ZOS}
       \vspace*{1\baselineskip}

       \vspace{0.75\baselineskip}

       {\LARGE\sc Souborový systém založený na i-uzlech\\}

       \vspace{4\baselineskip}
       
		\vspace{0.5\baselineskip}

       
       {\sc\Large Stanislav Král \\}

       \vspace{0.5\baselineskip}

       {A17B0260P}

       \vfill

       {\sc Západočeská univerzita v Plzni\\
       Fakulta aplikovaných věd}


    \end{titlepage}


    \tableofcontents
    \pagebreak


    \section{Popis hry Kris Kros}
    Jedná se o deskovou hru pro 2-4 hráče inspirovanou hrou Scrabble. Základním principem hry je skládat slova na herní desku z písmenek, která byla hráči rozdána. Hráči pokládají písmena na herní desku, kde buď v horizontálním vertikálním směru vznikají slova. Každé písmenko je ohodnoceno počtem bodů, které hráč za jeho položení dostane. Na konci každého kola se sečtou body písmenek všech slov, která v daném kole vznikla. Některá políčka na desce mohou bonifikovat celkové ohodnocení slova.

    \section{Analýza}
    \subsection{Návrh protokolu}
    	Protokol je třeba navrhnout podle charakteru aplikace, ve které bude použit. V případě hry Kris Kros jsou zprávy ve většině případů relativně jednoduché, ale například při aktualizaci herní desky, kdy je třeba poslat všem hráčům seznam ovlivněných políček, by taková datová část zprávy mohla být složitá a vlastní formát dat komplikovaný. To samé platí pro případ regenerace stavu klienta po výpadku připojení. V tomto případě jsou totiž data v datové části zprávy vrstvená a ideálně i popsaná nějakým identifikátorem. Návrh takového vlastního formátu by byl zbytečný, jelikož v dnešní době existuje mnoho standardizovaných formátů. V úvahu tedy připadá například XML nebo JSON.
   
	

    \section{Implementace}

		    \subsection{Struktura modulů}
			\subsubsection{aloc.go}
			Tento modul se stará o zápis dat do clusterů inodu. ZBěhem zápisu dat jsou kontinuálně alokovány clustery, do kterých bude zapisováno. Adresace, funguje tak, že dle pořadí právě zapisovaného clusteru se určí adresa pro zápis dat. Pokud je pořadí menší jak 5, tak se vybere přímý odkaz na data. Další odkazy jsou již nepřímé. Dále je zde implementována funkce \texttt{ReadDataFromInodeFx}, která umožňuje modulární práci se čtenými daty inodu.

			\subsubsection{bitmap.go}
			Poskytuje obecné funkce pro práci s bitmapou. Bitmapa je definována počáteční adresou bitmapy, její délkou v bytech a maximálním počtem bitů, které může obsahovat. Mezi důležitou funkci patří funkce \texttt{FindFreeBitsInBitmap}, která v bitmapě vyhledává volné bity. Při vyhledávání bitů se vyhledá více bitů než je potřeba, které se následně uloží do paměti a později se nemusí znovu přistupovat k souboru.
			\subsubsection{check.go}
			V tomto modulu jsou implementovány funkce pro ověření konzistence souborového systému. Kontola je prováděna dvěmi metodami. První metoda kontroluje, zdali je každý inode referencován nějakým jiným inodem (ověření, že každý soubor se nachází v nějaké složce). Druhá metoda ověřuje to, že udáváná velikost souboru koresponduje s počtem alokovaných clusterů

			\subsubsection{cluster.go}
			Pro práci s clustery jsou zde implementovány funkce, které jsou hojně využívány v celém programu. Umožňují například efektivně vyhledat volné ID clusteru, zapsat do clusteru adresu nebo například nějaký identifitkátor. Dále poskytuje funkce pro přěvod mezi ID a adresou clusteru.
			
			\subsubsection{commands.go}
			V tomto modulu jsou implementovány požadované příkazy souborového systému. Funkce v tomto modulu jsou většinou jednoduché a hojně využívají funkce ostatních modulů. Funkce pro interakci se skutečným souborovým systémem jsou složitější, jelikož pracují přímo se soubory.
						
			\subsubsection{dir.go}
			Modul, jež implementuje funkce pro práci s položkami adresáře. Nejsložitejší funkcí je funkce \texttt{AppendDirItem}, která počítá s tím, že může nastat situace, kdy je jedna položka adresáře zapsána do dvou clusterů. Dále se zde nacházejí funkce pro vyhledávání položek dle ID nebo názvu.
									
			\subsubsection{dir\_traversal.go}
O možnost navigovat mezi jednotlivými adresářemi souborového systému se stará právě tento modul. Poskytuje metody pro rozklíčování cesty na jednotlivé složky a pro změnu aktuálního adresáře. Za zmínku stojí funkce \texttt{VisitDirectoryByPathAndExecute}, která zajistí žmenu aktuálního adresáře na požadovaný, ve kterém zavolá příslušnou funkci, která byla předána parametrem. Nakonec se vrátí do původního adresáře, který byl aktuální před navigací. Většina implementovaných příkazů používá tuto funkci.

			\subsubsection{format.go}
Tento modul implementuje metody pro formátování souborového systému a vytvoření superbloku. V rámci formátování se právě vytváří nový superblok a kořenový adresář. Kořenový adresář se vytvoří pouze v případě, že se souborový systém nenachází v testovacím režimu, jelikož testy ověřující funkcionalitu programu počítají s tím, že všechny inody jsou volné.
					
			\subsubsection{id\_set.go}
Modul, který implementuje datovou strukturu \textit{množina} pro datový typ ID. Tato struktura se využívá při kontrole konzistence, kdy musí být jednotlivé položky adresáře unikátní.

			\subsubsection{inode.go}
Pro práci s inody slouží právě tento modul, který poskytuje funkce pro vyhledávání volného inodu, vytvoření nového inodu a pro převod mezi adresami a identifikátory inodů.

			\subsubsection{myfilesystem.go}
Definuje strukturu, jež představuje souborový systém. Dále poskytuje funkci \texttt{Load}, která se pokusí načíst souborový systém ze souboru. 

			\subsubsection{structs.go}
Definice základních struktur, které se používají napříč souborovým systémem. Nachází se zde například definice struktury reprezentující inod nebo cluster.


			
			\subsubsection{\textit{Keepalive} zprávy}
			Server vyžaduje, aby od každého klienta byly přijímány každé dvě sekundy zprávy typu \texttt{KeepAlive}. Pokud během dvou sekund nepřijde tato zpráva potvrzující funkčnost klienta, tak je spojení mezi klientem a serverem považováno za nevyhovující a klient je odpojen.
	
	Pokud se hráč znovu připojí a korektně se autorizuje, tak je jeho stav obnoven a hráč obdrží příslušnou zprávu, pomocí které zrekonstruje aktuální stav hry.
			
			\subsubsection{Nedodržování protokolu}
			Server vyžaduje, aby od každého klienta byly přijímány pouze zprávy splňující definovaný protokol. Pokud po sobě server přijme od nějakého klienta \texttt{5} nevalidných bytů, tak server klienta odpojí.
			\subsubsection{Použité knihovny}
			Pro logování byla použita knihovna \texttt{logrus} - \href{https://github.com/Sirupsen/logrus}{https://github.com/Sirupsen/logrus}
	    \subsection{Popis klienta}
	    Klient je napsaný v jazyce \textbf{Kotlin} a pro zobrazování GUI používá knihovnu \texttt{TornadoFX}, jenž obaluje knihovnu \texttt{JavaFX}.
	    Pro čtení ze socketu a periodické odesílání \textit{keepalive} zpráv jsou vytvořena speciální vlákna. Klient je tedy vícevláknový.
   		    \subsubsection{Struktura modulů}
   			\begin{itemize}

	    	\item \texttt{model} - definice použitých modelů
	    	\item \texttt{networking} - implementace definovaného protokolu, čtení ze socketů a definice zpráv implementace protokolu. Stejně jako na serveru, je čtení příchozích zpráv rozděleno do tří vrstev. Třída \texttt{Network} se stará o volání připojování k serveru a poskytuje metody, které slouží k registraci reakcí na příchozí zprávy.
 	    	\item \texttt{screens} - definice jednotlivých obrazovek aplikace. Ke každé obrazovce je ve zvláštním souboru definován \texttt{Controller} obsluhující události aplikace.
			\item \texttt{test} - obsahuje testy, které testují funkčnost parserů zpráv protokolu. Dále se zde také testují pomocné metody, jako například \texttt{IsNextByteEscaped}, která kontroluje, zdali by byl další byte escapovaný.
 	    	\end{itemize}
			\subsubsection{Nedodržování protokolu}
			Klient vyžaduje, aby od serveru byly přijímány pouze zprávy splňující definovaný protokol. Pokud po sobě kleint přijme od serveru \texttt{5} nevalidných bytů, tak klient přeruší se serverem spojení.
			\subsubsection{Použité knihovny}
			\begin{itemize}
				\item Pro logování byla použita knihovna \texttt{kotlin-logging} s využitím knihovny \texttt{logback} - \href{https://github.com/MicroUtils/kotlin-logging}{https://github.com/MicroUtils/kotlin-logging}. 

	\item Pro parsování JSON zpráv byla použita knihovna \texttt{klaxon} -\\ \href{https://github.com/cbeust/klaxon}{https://github.com/cbeust/klaxon}. 
		\item Pro testování byla využita knihovna \texttt{JUnit 5.5.2}
			\end{itemize}
			
	\section{Překlad a spuštění aplikace}

	V kořenu projektu klienta přeložte a spusťte klienta pomocí příkazu \texttt{./gradlew run}.

    \section{Závěr}
    V rámci této semestrální práce byl vytvořena dvojice programů - server a klient napodující deskovou hru Kris Kros. Důkladný a pečlivý návrh protokolu se osvědčil a v průběhu vývoje neprošel žádnými změnami. Avšak co se několikrát měnilo, byl způsob čtení zpráv, kdy nakonec jsou zprávy čteny spolehlivým automatem. Pokud bych automat navrhl hned na začátku, mohl jsem si ušetřit spoustu času laděním a upravováním parseru zpráv. Aplikace i server jsou však nyní funkční a hra je hratelná. Pokud je spojení s některým hráčem dočasně nedostupné nebo je klient násilně vypnut, tak po autorizaci je hráčovo stav korektně obnoven.
    


%obrazek
%\begin{figure}[!ht]
%\centering
%{\includegraphics[width=12cm]{img/poly-example.jpeg}}
%\caption{Zjednodušené UML aplikace (pouze balíčky)}
%\label{fig:photo}
%\end{figure}

	
	

\end{document}    